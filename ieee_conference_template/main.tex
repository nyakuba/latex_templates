% !TEX encoding = UTF-8 Unicode
%----------------------------------------------------------------------------------------
\documentclass[conference]{IEEEtran}
% If IEEEtran.cls has not been installed into the LaTeX system files,
% manually specify the path to it like:
% \documentclass[conference]{../sty/IEEEtran}

%----------------------------------------------------------------------------------------
% Heiko Oberdiek's ifpdf.sty is very useful if you need conditional
% compilation based on whether the output is pdf or dvi.
% usage:
% \ifpdf
%   % pdf code
% \else
%   % dvi code
% \fi
% The latest version of ifpdf.sty can be obtained from:
% http://www.ctan.org/pkg/ifpdf
% When switching from latex to pdflatex and vice-versa, the compiler may
% have to be run twice to clear warning/error messages.
%\usepackage{ifpdf}

%----------------------------------------------------------------------------------------
% The package provides an easy and flexible user interface to customize page layout,
% implementing auto-centering and auto-balancing mechanisms so that the users have only
% to give the least description for the page layout.
%\usepackage[margin=10mm]{geometry}
% Introduces command \newgeometry{left=10mm,right=30mm,top=20mm,bottom=20mm},
% that can be used to change margins in the middle of the document.

%----------------------------------------------------------------------------------------
% Fontspec is a package for XeLaTeX and LuaLaTeX.
% It provides an automatic and unified interface to feature-rich AAT and OpenType fonts
% through the NFSS in LaTeX running on XeTeX or LuaTeX engines.
\usepackage{fontspec}

% Windows standard fonts with cyrillic support:
%\setmainfont{Times New Roman}
%\setsansfont{Arial}
%\setmonofont{Latin Modern Mono}

% Linux standard fonts with cyrillic support:
%\setmainfont{Liberation Serif}
%\setsansfont{Liberation Sans}
%\setmonofont{Liberation Mono}

% It is also possible to use local fonts:
\setmainfont[Path=./fonts/, Extension=.ttf,
             UprightFont=*-Regular,
             ItalicFont=*-Italic,
             BoldFont=*-Bold,
             BoldItalicFont=*-BoldItalic]{LiberationSerif}
\setsansfont[Path=./fonts/, Extension=.ttf,
             UprightFont=*-Regular,
             ItalicFont=*-Italic,
             BoldFont=*-Bold,
             BoldItalicFont=*-BoldItalic]{LiberationSans}
\setmonofont[Path=./fonts/, Extension=.ttf,
             UprightFont=*-Regular,
             ItalicFont=*-Italic,
             BoldFont=*-Bold,
             BoldItalicFont=*-BoldItalic]{LiberationMono}

%----------------------------------------------------------------------------------------
% Provides support for setting the spacing between lines in a document.
% Package options include singlespacing, onehalfspacing, and doublespacing.
% Introduces \singlespacing, \onehalfspacing, and \doublespacing commands.
% Other size spacings also available (\setstretch). 
\usepackage{setspace}

%----------------------------------------------------------------------------------------
% Alternative: babel package
% This package manages culturally-determined typographical (and other) rules
% for a wide range of languages. A document may select a single language to be supported,
% or it may select several, in which case the document may switch
% from one language to another in a variety of ways.
%\usepackage{polyglossia}
%\setmainlanguage{russian}
%\setotherlanguage{english}
% Note, that since polyglossia hardly relies on fontspec package,
% defined above fonts should support chosen language.
% Otherwise you will get a polyglossia error.

% Use \selectlanguage{<language>} to switch between languages, and
% \foreignlanguage{<language>}{<text>} to write text on specified language.
% Define auxiliary commands to switch languages.
%\newcommand{\ru}[1]{\foreignlanguage{russian}{#1}}
%\newcommand{\en}[1]{\foreignlanguage{english}{#1}}

%----------------------------------------------------------------------------------------
% Alternative: polyglossia package
% This package manages culturally-determined typographical (and other) rules
% for a wide range of languages. A document may select a single language to be supported,
% or it may select several, in which case the document may switch
% from one language to another in a variety of ways.
\usepackage[main=english,russian]{babel}
% Note, that since babel hardly relies on font configuration,
% defined above fonts should support chosen language.
% Otherwise you will get a polyglossia error.

% Use \selectlanguage{<language>} to switch between languages, and
% \foreignlanguage{<language>}{<text>} to write text on specified language.
% Define auxiliary commands to switch languages.
\newcommand{\ru}[1]{\foreignlanguage{russian}{#1}}
\newcommand{\en}[1]{\foreignlanguage{english}{#1}}

%----------------------------------------------------------------------------------------
% Bibliography package
\usepackage[
  backend=biber,       % use biber to process .bib files
  style=ieee,
  autolang=other,
  isbn=true,
  url=false
]{biblatex}

\bibliography{IEEEabrv,bibliography.bib}

%----------------------------------------------------------------------------------------
% Implements a command that causes the commands specified in its argument to be expanded
% after the current page is output. Now you can issue \afterpage{\clearpage}
% and the current page will be filled up with text as usual, but then the \clearpage command
% will flush out all the deferred floats before the next text page begins.
\usepackage{afterpage}

%----------------------------------------------------------------------------------------
% This package provides user control over the layout of the three basic list environments:
% enumerate, itemize and description.
\usepackage{enumitem}

% Example: \begin{enumerate}[label=\roman*.] --- use roman numbers followed by dot in items

%----------------------------------------------------------------------------------------
\usepackage{amsmath,amsthm,amssymb}  % math packages
\usepackage{dsfont}  % math font

%----------------------------------------------------------------------------------------
% The caption package provides many ways to customise the captions in floating environments
% like figure and table, and cooperates with many other packages
\usepackage{caption}

%----------------------------------------------------------------------------------------
% This package provides advanced facilities for inline and display quotations.
% Recommended for use with babel/polyglossia
\usepackage{csquotes}
% Introduces commands:
% \blockquote[<cite>][<punct>]{<text>} --- quote text
% \hyphenblockquote[<cite>][<punct>]{<text>} --- quote text on a foreign language

%----------------------------------------------------------------------------------------
% The package provides extensive facilities, both for constructing headers and footers,
% and for controlling their use.
\usepackage{fancyhdr}
\pagestyle{fancy}

% Uncomment this block to аdd link to table of contents at the footer of each page
%\renewcommand{\headrulewidth}{0pt}% removes header line
%\fancypagestyle{plain}{% for chapter starting pages
%  \fancyhf{}  % clears header fields
%  \cfoot{\hyperlink{contents}{\contentsname}\hfill\thepage}}
%\cfoot{\hyperlink{contents}{\contentsname}\hfill\thepage}  % links TOC to page footer

%----------------------------------------------------------------------------------------
% Provides variants of \fbox: \shadowbox, \doublebox, \ovalbox, \Ovalbox, with helpful tools
% for using box macros and flexible verbatim macros.
% You can box mathematics, floats, center, flushleft, and flushright, lists, and pages.
\usepackage{fancybox}

%----------------------------------------------------------------------------------------
% This package prevents page numbers and headings from appearing on empty pages. 
\usepackage{emptypage}

%----------------------------------------------------------------------------------------
% The appendix package provides various ways of formatting the titles of appendices. 
%\usepackage{appendix}

%----------------------------------------------------------------------------------------
% The verbatim package reimplements the LaTeX verbatim and verbatim* environments.
% The package also provides a comment environment (that skips everything between
% \begin{comment} and \end{comment}), and a command \verbatiminput for typesetting
% the contents of a file, verbatim.
\usepackage{verbatim}

%----------------------------------------------------------------------------------------
% The package was designed to accommodate all needs for inclusion of graphics in LaTeX documents
\usepackage{graphicx}

% Declare paths to graphics and extensions
\graphicspath{{./pictures/}{./plots/eps}}
\DeclareGraphicsExtensions{.pdf,.jpeg,.png,.eps}

%----------------------------------------------------------------------------------------
% epstopdf is a Perl script that converts an EPS file to an 'encapsulated' PDF file
\usepackage{epstopdf}
\epstopdfsetup{outdir=./build/}  % set output directory for auxiliary files
\epstopdfsetup{update}  % only regenerate pdf files when eps file is newer

%----------------------------------------------------------------------------------------
% The color package provides both foreground (text, rules, etc.) and background colour management
% Relies on graphicx package
\usepackage{xcolor}

%----------------------------------------------------------------------------------------
% An extended implementation of the array and tabular environments which extends the options for column formats,
% and provides "programmable" format specifications.
\usepackage{array}

%----------------------------------------------------------------------------------------
% These packages offer a series of extensions to the standard tabular environment:
% multirow provides a construction for table cells that span more than one row of the table;
% bigstrut creates struts which (slightly) stretch the table row in which they sit;
% bigdelim creates an appropriately-sized delimiter (for example, brace, parenthesis or bracket)
% to fit in a single multirow, to indicate a relationship between other rows
\usepackage{multirow}

%----------------------------------------------------------------------------------------
% The package enhances the quality of tables in LaTeX, providing extra commands
% as well as behind-the-scenes optimisation.
% Allows the use of \toprule, \midrule and \bottomrule commands in tables.
\usepackage{booktabs}

%----------------------------------------------------------------------------------------
% Provides support for the manipulation and reference of small or ‘sub’ figures and tables
% within a single figure or table environment
\usepackage[caption=false,font=normalsize,labelfont=sf,textfont=sf]{subfig}

%----------------------------------------------------------------------------------------
% Multicol defines a multicols environment which typesets text in multiple columns
% (up to a maximum of 10), and (by default) balances the end of each column at the end of the environment.
% To adjust column contents more precisely use minipage environment.
\usepackage{multicol}

%----------------------------------------------------------------------------------------
% The hyperref package is used to handle cross-referencing
% commands in LaTeX to produce hypertext links in the document.
\usepackage{hyperref}

% Define color scheme of links (xcolor package is required)
\hypersetup{
  unicode,
  colorlinks,
  citecolor=red,
  filecolor=black,
  linkcolor=red,
  urlcolor=red
}

%----------------------------------------------------------------------------------------
% The command \url is a form of verbatim command that allows linebreaks at certain
% characters or combinations of characters, accepts reconfiguration,
% and can usually be used in the argument to another command.
\usepackage{url}
\urlstyle{same}  % use current text font for url links

%----------------------------------------------------------------------------------------
% The package enables the user to typeset programs (programming code) within LaTeX;
% the source code is read directly by TeX --- no frontend processor is needed.
\usepackage{listings}

% Define styles for C++ listings
\definecolor{dkgreen}{rgb}{0,0.6,0}
\definecolor{gray}{rgb}{0.5,0.5,0.5}
\definecolor{mauve}{rgb}{0.58,0,0.82}

\lstset{frame=tb,
  language=C++,
  aboveskip=3mm,
  belowskip=3mm,
  showstringspaces=false,
  columns=flexible,
  basicstyle={\small\ttfamily},
  numbers=left,
  numberstyle=\tiny\color{gray},
  keywordstyle=\color{blue},
  commentstyle=\color{dkgreen},
  stringstyle=\color{mauve},
  breaklines=true,
  breakatwhitespace=true,
  tabsize=3,
  xleftmargin=0.5cm,
  frame=lr,
  framesep=8pt,
  framerule=0pt,
  otherkeywords={*,__m256i}
}

%----------------------------------------------------------------------------------------
% Algorithm2e provides an environment for writing algorithms
\usepackage[ruled,lined,noend,linesnumbered]{algorithm2e}

%----------------------------------------------------------------------------------------
% PGF is a macro package for creating graphics
\usepackage{tikz}
%\usepackage{pgffor}
%\usepackage{ifthen}
%\usepackage{animate}

%----------------------------------------------------------------------------------------
% This package provides \todo{} command, that places visual notifications in text
% use [disable] option to remove all todoes from text
\usepackage[colorinlistoftodos,prependcaption,textsize=tiny]{todonotes}

\presetkeys{todonotes}{inline}{}  % set inline option for \todo command by default
\setlength{\marginparwidth}{2cm}  % set width of margin where comment will be placed

%----------------------------------------------------------------------------------------
%       GLOBAL CONFIGURATION AND DEFINITIONS
%----------------------------------------------------------------------------------------
% Declare math operators such as \min or \arg that will be used in math formulas
\DeclareMathOperator{\argmax}{argmax}
\DeclareMathOperator{\diag}{diag}
\DeclareMathOperator{\GF}{GF}
\DeclareMathOperator{\sgn}{sgn}
\DeclareMathOperator{\wt}{wt}

%----------------------------------------------------------------------------------------
\newtheorem{mdefinition}{Определение}
\newtheorem{mtheorem}{Теорема}
\newtheorem{mexample}{Пример}

%----------------------------------------------------------------------------------------
%       CUSTOM COMMANDS
%----------------------------------------------------------------------------------------
% Define shortcuts
\newcommand\mbA{\mathbb{A}} \newcommand\mbB{\mathbb{B}}
\newcommand\mbC{\mathbb{C}} \newcommand\mbD{\mathbb{D}}
\newcommand\mbE{\mathbb{E}} \newcommand\mbF{\mathbb{F}}
\newcommand\mbG{\mathbb{G}} \newcommand\mbH{\mathbb{H}}
\newcommand\mbI{\mathbb{I}} \newcommand\mbJ{\mathbb{J}}
\newcommand\mbK{\mathbb{K}} \newcommand\mbL{\mathbb{L}}
\newcommand\mbM{\mathbb{M}} \newcommand\mbN{\mathbb{N}}
\newcommand\mbO{\mathbb{0}} \newcommand\mbP{\mathbb{P}}
\newcommand\mbQ{\mathbb{Q}} \newcommand\mbR{\mathbb{R}}
\newcommand\mbS{\mathbb{S}} \newcommand\mbT{\mathbb{T}}
\newcommand\mbV{\mathbb{V}} \newcommand\mbU{\mathbb{U}}
\newcommand\mbW{\mathbb{W}} \newcommand\mbX{\mathbb{X}}
\newcommand\mbY{\mathbb{Y}} \newcommand\mbZ{\mathbb{Z}}

\newcommand\mcA{\mathcal{A}} \newcommand\mcB{\mathcal{B}}
\newcommand\mcC{\mathcal{C}} \newcommand\mcD{\mathcal{D}}
\newcommand\mcE{\mathcal{E}} \newcommand\mcF{\mathcal{F}}
\newcommand\mcG{\mathcal{G}} \newcommand\mcH{\mathcal{H}}
\newcommand\mcI{\mathcal{I}} \newcommand\mcJ{\mathcal{J}}
\newcommand\mcK{\mathcal{K}} \newcommand\mcL{\mathcal{L}}
\newcommand\mcM{\mathcal{M}} \newcommand\mcN{\mathcal{N}}
\newcommand\mcO{\mathcal{0}} \newcommand\mcP{\mathcal{P}}
\newcommand\mcQ{\mathcal{Q}} \newcommand\mcR{\mathcal{R}}
\newcommand\mcS{\mathcal{S}} \newcommand\mcT{\mathcal{T}}
\newcommand\mcV{\mathcal{V}} \newcommand\mcU{\mathcal{U}}
\newcommand\mcW{\mathcal{W}} \newcommand\mcX{\mathcal{X}}
\newcommand\mcY{\mathcal{Y}} \newcommand\mcZ{\mathcal{Z}}

%----------------------------------------------------------------------------------------
% \newcommand* cannot contain \par
\newcommand*\norm[1]{\left\lVert#1\right\rVert}
\newcommand*\set[1]{\{#1\}}
\newcommand*\tuple[1]{\langle #1 \rangle}

% put text into a circle
\newcommand*\circled[1]{\tikz[baseline=(char.base)]{
            \node[shape=circle,draw,inner sep=2pt] (char) {#1};}}
% put text in a square
\newcommand*\squared[1]{\tikz[baseline=(char.base)]{
            \node[shape=rectangle,draw,inner sep=3pt] (char) {#1};}}

%----------------------------------------------------------------------------------------
%       DOCUMENT
%----------------------------------------------------------------------------------------
% correct bad hyphenation here
\hyphenation{op-tical net-works semi-conduc-tor}

%----------------------------------------------------------------------------------------
\begin{document}
%
% paper title
% Titles are generally capitalized except for words such as a, an, and, as,
% at, but, by, for, in, nor, of, on, or, the, to and up, which are usually
% not capitalized unless they are the first or last word of the title.
% Linebreaks \\ can be used within to get better formatting as desired.
% Do not put math or special symbols in the title.
\title{Bare Demo of IEEEtran.cls\\ for IEEE Conferences}

% author names and affiliations
% use a multiple column layout for up to three different
% affiliations
\author{\IEEEauthorblockN{Michael Shell}
\IEEEauthorblockA{School of Electrical and\\Computer Engineering\\
Georgia Institute of Technology\\
Atlanta, Georgia 30332--0250\\
Email: http://www.michaelshell.org/contact.html}
\and
\IEEEauthorblockN{Homer Simpson}
\IEEEauthorblockA{Twentieth Century Fox\\
Springfield, USA\\
Email: homer@thesimpsons.com}
\and
\IEEEauthorblockN{James Kirk\\ and Montgomery Scott}
\IEEEauthorblockA{Starfleet Academy\\
San Francisco, California 96678--2391\\
Telephone: (800) 555--1212\\
Fax: (888) 555--1212}}

% conference papers do not typically use \thanks and this command
% is locked out in conference mode. If really needed, such as for
% the acknowledgment of grants, issue a \IEEEoverridecommandlockouts
% after \documentclass

% for over three affiliations, or if they all won't fit within the width
% of the page, use this alternative format:
% 
%\author{\IEEEauthorblockN{Michael Shell\IEEEauthorrefmark{1},
%Homer Simpson\IEEEauthorrefmark{2},
%James Kirk\IEEEauthorrefmark{3}, 
%Montgomery Scott\IEEEauthorrefmark{3} and
%Eldon Tyrell\IEEEauthorrefmark{4}}
%\IEEEauthorblockA{\IEEEauthorrefmark{1}School of Electrical and Computer Engineering\\
%Georgia Institute of Technology,
%Atlanta, Georgia 30332--0250\\ Email: see http://www.michaelshell.org/contact.html}
%\IEEEauthorblockA{\IEEEauthorrefmark{2}Twentieth Century Fox, Springfield, USA\\
%Email: homer@thesimpsons.com}
%\IEEEauthorblockA{\IEEEauthorrefmark{3}Starfleet Academy, San Francisco, California 96678-2391\\
%Telephone: (800) 555--1212, Fax: (888) 555--1212}
%\IEEEauthorblockA{\IEEEauthorrefmark{4}Tyrell Inc., 123 Replicant Street, Los Angeles, California 90210--4321}}

% use for special paper notices
%\IEEEspecialpapernotice{(Invited Paper)}

% make the title area
\maketitle

%----------------------------------------------------------------------------------------
% As a general rule, do not put math, special symbols or citations
% in the abstract
\begin{abstract}
The abstract goes here.
\end{abstract}

% no keywords

% For peer review papers, you can put extra information on the cover
% page as needed:
% \ifCLASSOPTIONpeerreview
% \begin{center} \bfseries EDICS Category: 3-BBND \end{center}
% \fi
%
% For peerreview papers, this IEEEtran command inserts a page break and
% creates the second title. It will be ignored for other modes.
\IEEEpeerreviewmaketitle

%----------------------------------------------------------------------------------------
\section{Introduction}
% no \IEEEPARstart
This demo file is intended to serve as a ``starter file''
for IEEE conference papers produced under \LaTeX\ using
IEEEtran.cls version 1.8b and later.
% You must have at least 2 lines in the paragraph with the drop letter
% (should never be an issue)
I wish you the best of success.

\hfill mds
 
\hfill August 26, 2015

%----------------------------------------------------------------------------------------
\subsection{Subsection Heading Here}
Subsection text here.

%----------------------------------------------------------------------------------------
\subsubsection{Subsubsection Heading Here}
Subsubsection text here.

% An example of a floating figure using the graphicx package.
% Note that \label must occur AFTER (or within) \caption.
% For figures, \caption should occur after the \includegraphics.
% Note that IEEEtran v1.7 and later has special internal code that
% is designed to preserve the operation of \label within \caption
% even when the captionsoff option is in effect. However, because
% of issues like this, it may be the safest practice to put all your
% \label just after \caption rather than within \caption{}.
%
% Reminder: the "draftcls" or "draftclsnofoot", not "draft", class
% option should be used if it is desired that the figures are to be
% displayed while in draft mode.
%
%\begin{figure}[!t]
%\centering
%\includegraphics[width=2.5in]{myfigure}
% where an .eps filename suffix will be assumed under latex, 
% and a .pdf suffix will be assumed for pdflatex; or what has been declared
% via \DeclareGraphicsExtensions.
%\caption{Simulation results for the network.}
%\label{fig_sim}
%\end{figure}

% Note that the IEEE typically puts floats only at the top, even when this
% results in a large percentage of a column being occupied by floats.


% An example of a double column floating figure using two subfigures.
% (The subfig.sty package must be loaded for this to work.)
% The subfigure \label commands are set within each subfloat command,
% and the \label for the overall figure must come after \caption.
% \hfil is used as a separator to get equal spacing.
% Watch out that the combined width of all the subfigures on a 
% line do not exceed the text width or a line break will occur.
%
%\begin{figure*}[!t]
%\centering
%\subfloat[Case I]{\includegraphics[width=2.5in]{box}%
%\label{fig_first_case}}
%\hfil
%\subfloat[Case II]{\includegraphics[width=2.5in]{box}%
%\label{fig_second_case}}
%\caption{Simulation results for the network.}
%\label{fig_sim}
%\end{figure*}
%
% Note that often IEEE papers with subfigures do not employ subfigure
% captions (using the optional argument to \subfloat[]), but instead will
% reference/describe all of them (a), (b), etc., within the main caption.
% Be aware that for subfig.sty to generate the (a), (b), etc., subfigure
% labels, the optional argument to \subfloat must be present. If a
% subcaption is not desired, just leave its contents blank,
% e.g., \subfloat[].


% An example of a floating table. Note that, for IEEE style tables, the
% \caption command should come BEFORE the table and, given that table
% captions serve much like titles, are usually capitalized except for words
% such as a, an, and, as, at, but, by, for, in, nor, of, on, or, the, to
% and up, which are usually not capitalized unless they are the first or
% last word of the caption. Table text will default to \footnotesize as
% the IEEE normally uses this smaller font for tables.
% The \label must come after \caption as always.
%
%\begin{table}[!t]
%% increase table row spacing, adjust to taste
%\renewcommand{\arraystretch}{1.3}
% if using array.sty, it might be a good idea to tweak the value of
% \extrarowheight as needed to properly center the text within the cells
%\caption{An Example of a Table}
%\label{table_example}
%\centering
%% Some packages, such as MDW tools, offer better commands for making tables
%% than the plain LaTeX2e tabular which is used here.
%\begin{tabular}{|c||c|}
%\hline
%One & Two\\
%\hline
%Three & Four\\
%\hline
%\end{tabular}
%\end{table}


% Note that the IEEE does not put floats in the very first column
% - or typically anywhere on the first page for that matter. Also,
% in-text middle ("here") positioning is typically not used, but it
% is allowed and encouraged for Computer Society conferences (but
% not Computer Society journals). Most IEEE journals/conferences use
% top floats exclusively. 
% Note that, LaTeX2e, unlike IEEE journals/conferences, places
% footnotes above bottom floats. This can be corrected via the
% \fnbelowfloat command of the stfloats package.

%----------------------------------------------------------------------------------------
\section{Conclusion}
The conclusion goes here.

% conference papers do not normally have an appendix

%----------------------------------------------------------------------------------------
% use section* for acknowledgment
\section*{Acknowledgment}

The authors would like to thank...

% trigger a \newpage just before the given reference
% number - used to balance the columns on the last page
% adjust value as needed - may need to be readjusted if
% the document is modified later
%\IEEEtriggeratref{8}
% The "triggered" command can be changed if desired:
%\IEEEtriggercmd{\enlargethispage{-5in}}

%----------------------------------------------------------------------------------------
% references section
% can use a bibliography generated by BibTeX as a .bbl file
%\printbibliography

\begin{thebibliography}{1}

\bibitem{IEEEhowto:kopka}
H.~Kopka and P.~W. Daly, \emph{A Guide to \LaTeX}, 3rd~ed.\hskip 1em plus
  0.5em minus 0.4em\relax Harlow, England: Addison-Wesley, 1999.

\end{thebibliography}

% that's all folks
\end{document}


